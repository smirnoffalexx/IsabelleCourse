%
\begin{isabellebody}%
\setisabellecontext{Locales}%
%
\isadelimtheory
%
\endisadelimtheory
%
\isatagtheory
\isacommand{theory}\isamarkupfalse%
\ Locales\isanewline
\ \ \isakeyword{imports}\ {\isachardoublequoteopen}HOL{\isacharminus}{\kern0pt}Library{\isachardot}{\kern0pt}Countable{\isacharunderscore}{\kern0pt}Set{\isachardoublequoteclose}\ \isanewline
\isakeyword{begin}%
\endisatagtheory
{\isafoldtheory}%
%
\isadelimtheory
%
\endisadelimtheory
%
\isadelimdocument
%
\endisadelimdocument
%
\isatagdocument
%
\isamarkupsection{Locales%
}
\isamarkuptrue%
%
\endisatagdocument
{\isafolddocument}%
%
\isadelimdocument
%
\endisadelimdocument
%
\begin{isamarkuptext}%
Locales are based on context . 
A context is a formula scheme  \isa{{\isasymAnd}\ x\isactrlsub {\isadigit{1}}\ {\isachardot}{\kern0pt}\ {\isachardot}{\kern0pt}\ {\isachardot}{\kern0pt}\ x\isactrlsub n\ {\isachardot}{\kern0pt}\ {\isacharbrackleft}{\kern0pt}{\isacharbrackleft}{\kern0pt}\ A\isactrlsub {\isadigit{1}}\ {\isacharsemicolon}{\kern0pt}\ {\isachardot}{\kern0pt}\ {\isachardot}{\kern0pt}\ {\isachardot}{\kern0pt}\ {\isacharsemicolon}{\kern0pt}A\isactrlsub m\ {\isacharbrackright}{\kern0pt}{\isacharbrackright}{\kern0pt}\ {\isasymLongrightarrow}\ {\isachardot}{\kern0pt}\ {\isachardot}{\kern0pt}\ {\isachardot}{\kern0pt}}%
\end{isamarkuptext}\isamarkuptrue%
\isacommand{locale}\isamarkupfalse%
\ partial{\isacharunderscore}{\kern0pt}order\ {\isacharequal}{\kern0pt}\isanewline
\ \ \isakeyword{fixes}\ le\ {\isacharcolon}{\kern0pt}{\isacharcolon}{\kern0pt}\ {\isachardoublequoteopen}{\isacharprime}{\kern0pt}a\ {\isasymRightarrow}\ {\isacharprime}{\kern0pt}a\ {\isasymRightarrow}\ bool{\isachardoublequoteclose}\ {\isacharparenleft}{\kern0pt}\isakeyword{infixl}\ {\isachardoublequoteopen}{\isasymsqsubseteq}{\isachardoublequoteclose}\ {\isadigit{5}}{\isadigit{0}}{\isacharparenright}{\kern0pt}\isanewline
\ \ \isakeyword{assumes}\ refl\ {\isacharbrackleft}{\kern0pt}intro{\isacharcomma}{\kern0pt}\ simp{\isacharbrackright}{\kern0pt}{\isacharcolon}{\kern0pt}\ {\isachardoublequoteopen}x\ {\isasymsqsubseteq}\ x{\isachardoublequoteclose}\isanewline
\ \ \ \ \isakeyword{and}\ anti{\isacharunderscore}{\kern0pt}sym\ {\isacharbrackleft}{\kern0pt}intro{\isacharbrackright}{\kern0pt}{\isacharcolon}{\kern0pt}\ {\isachardoublequoteopen}{\isasymlbrakk}\ x\ {\isasymsqsubseteq}\ y{\isacharsemicolon}{\kern0pt}\ y\ {\isasymsqsubseteq}\ x\ {\isasymrbrakk}\ {\isasymLongrightarrow}\ x\ {\isacharequal}{\kern0pt}\ y{\isachardoublequoteclose}\isanewline
\ \ \ \ \isakeyword{and}\ trans\ {\isacharbrackleft}{\kern0pt}trans{\isacharbrackright}{\kern0pt}{\isacharcolon}{\kern0pt}\ {\isachardoublequoteopen}{\isasymlbrakk}\ x\ {\isasymsqsubseteq}\ y{\isacharsemicolon}{\kern0pt}\ y\ {\isasymsqsubseteq}\ z\ {\isasymrbrakk}\ {\isasymLongrightarrow}\ x\ {\isasymsqsubseteq}\ z{\isachardoublequoteclose}%
\begin{isamarkuptext}%
The parameter of this locale is le, which is a binary predicate with infix syntax \isa{{\isasymsqsubseteq}}. 
The parameter syntax is available in the subsequent assumptions,
which are the familiar partial order axioms.%
\end{isamarkuptext}\isamarkuptrue%
\isacommand{print{\isacharunderscore}{\kern0pt}locales}\isamarkupfalse%
\isanewline
\isanewline
\isacommand{print{\isacharunderscore}{\kern0pt}locale}\isamarkupfalse%
{\isacharbang}{\kern0pt}\ partial{\isacharunderscore}{\kern0pt}order%
\begin{isamarkuptext}%
The assumptions have turned into conclusions, denoted by the keyword notes. 
Also, there is only one assumption - \isa{partial{\isacharunderscore}{\kern0pt}order}. 
The locale declaration has introduced the predicate \isa{partial{\isacharunderscore}{\kern0pt}order} to the theory. 
This predicate is the locale predicate.%
\end{isamarkuptext}\isamarkuptrue%
\isacommand{thm}\isamarkupfalse%
\ partial{\isacharunderscore}{\kern0pt}order{\isacharunderscore}{\kern0pt}def\isanewline
\isacommand{thm}\isamarkupfalse%
\ partial{\isacharunderscore}{\kern0pt}order{\isachardot}{\kern0pt}trans\ partial{\isacharunderscore}{\kern0pt}order{\isachardot}{\kern0pt}anti{\isacharunderscore}{\kern0pt}sym\ partial{\isacharunderscore}{\kern0pt}order{\isachardot}{\kern0pt}refl%
\isadelimdocument
%
\endisadelimdocument
%
\isatagdocument
%
\isamarkupsubsection{Extending Locales%
}
\isamarkuptrue%
%
\endisatagdocument
{\isafolddocument}%
%
\isadelimdocument
%
\endisadelimdocument
\isacommand{definition}\isamarkupfalse%
\ {\isacharparenleft}{\kern0pt}\isakeyword{in}\ partial{\isacharunderscore}{\kern0pt}order{\isacharparenright}{\kern0pt}\isanewline
less\ {\isacharcolon}{\kern0pt}{\isacharcolon}{\kern0pt}\ {\isachardoublequoteopen}{\isacharprime}{\kern0pt}a\ {\isasymRightarrow}\ {\isacharprime}{\kern0pt}a\ {\isasymRightarrow}\ bool{\isachardoublequoteclose}\ {\isacharparenleft}{\kern0pt}\isakeyword{infixl}\ {\isachardoublequoteopen}{\isasymsqsubset}{\isachardoublequoteclose}\ {\isadigit{5}}{\isadigit{0}}{\isacharparenright}{\kern0pt}\ \isanewline
\ \ \isakeyword{where}\ {\isachardoublequoteopen}{\isacharparenleft}{\kern0pt}x\ {\isasymsqsubset}\ y{\isacharparenright}{\kern0pt}\ {\isacharequal}{\kern0pt}\ {\isacharparenleft}{\kern0pt}x\ {\isasymsqsubseteq}\ y\ {\isasymand}\ x\ {\isasymnoteq}\ y{\isacharparenright}{\kern0pt}{\isachardoublequoteclose}%
\begin{isamarkuptext}%
The definition generates a foundational constant \isa{partial{\isacharunderscore}{\kern0pt}order{\isachardot}{\kern0pt}less{\isacharcolon}{\kern0pt}}%
\end{isamarkuptext}\isamarkuptrue%
\isacommand{thm}\isamarkupfalse%
\ partial{\isacharunderscore}{\kern0pt}order{\isachardot}{\kern0pt}less{\isacharunderscore}{\kern0pt}def%
\begin{isamarkuptext}%
At the same time, the locale is extended by syntax transformations hiding
this construction in the context of the locale.%
\end{isamarkuptext}\isamarkuptrue%
\isacommand{print{\isacharunderscore}{\kern0pt}locale}\isamarkupfalse%
{\isacharbang}{\kern0pt}\ partial{\isacharunderscore}{\kern0pt}order\isanewline
\isanewline
\isacommand{lemma}\isamarkupfalse%
\ {\isacharparenleft}{\kern0pt}\isakeyword{in}\ partial{\isacharunderscore}{\kern0pt}order{\isacharparenright}{\kern0pt}\ less{\isacharunderscore}{\kern0pt}le{\isacharunderscore}{\kern0pt}trans\ {\isacharbrackleft}{\kern0pt}trans{\isacharbrackright}{\kern0pt}{\isacharcolon}{\kern0pt}\isanewline
\ \ {\isachardoublequoteopen}{\isasymlbrakk}\ x\ {\isasymsqsubset}\ y{\isacharsemicolon}{\kern0pt}\ y\ {\isasymsqsubseteq}\ z\ {\isasymrbrakk}\ {\isasymLongrightarrow}\ x\ {\isasymsqsubset}\ z{\isachardoublequoteclose}\isanewline
%
\isadelimproof
\ \ %
\endisadelimproof
%
\isatagproof
\isacommand{unfolding}\isamarkupfalse%
\ less{\isacharunderscore}{\kern0pt}def\ \isacommand{by}\isamarkupfalse%
\ {\isacharparenleft}{\kern0pt}blast\ intro{\isacharcolon}{\kern0pt}\ trans{\isacharparenright}{\kern0pt}%
\endisatagproof
{\isafoldproof}%
%
\isadelimproof
%
\endisadelimproof
%
\isadelimdocument
%
\endisadelimdocument
%
\isatagdocument
%
\isamarkupsubsection{context n begin ... end%
}
\isamarkuptrue%
%
\endisatagdocument
{\isafolddocument}%
%
\isadelimdocument
%
\endisadelimdocument
%
\begin{isamarkuptext}%
Entering locale context:%
\end{isamarkuptext}\isamarkuptrue%
\isacommand{context}\isamarkupfalse%
\ partial{\isacharunderscore}{\kern0pt}order\isanewline
\isanewline
\isakeyword{begin}\isanewline
\isanewline
\isacommand{definition}\isamarkupfalse%
\isanewline
\ \ is{\isacharunderscore}{\kern0pt}inf\ \isakeyword{where}\ {\isachardoublequoteopen}is{\isacharunderscore}{\kern0pt}inf\ x\ y\ i\ {\isacharequal}{\kern0pt}\isanewline
\ \ \ \ \ \ \ \ \ {\isacharparenleft}{\kern0pt}i\ {\isasymsqsubseteq}\ x\ {\isasymand}\ i\ {\isasymsqsubseteq}\ y\ {\isasymand}\ {\isacharparenleft}{\kern0pt}{\isasymforall}\ z{\isachardot}{\kern0pt}\ z\ {\isasymsqsubseteq}\ x\ {\isasymand}\ z\ {\isasymsqsubseteq}\ y\ {\isasymlongrightarrow}\ z\ {\isasymsqsubseteq}\ i{\isacharparenright}{\kern0pt}{\isacharparenright}{\kern0pt}{\isachardoublequoteclose}\isanewline
\isanewline
\isacommand{definition}\isamarkupfalse%
\isanewline
\ \ is{\isacharunderscore}{\kern0pt}sup\ \isakeyword{where}\ {\isachardoublequoteopen}is{\isacharunderscore}{\kern0pt}sup\ x\ y\ s\ {\isacharequal}{\kern0pt}\isanewline
\ \ \ \ \ \ \ \ \ {\isacharparenleft}{\kern0pt}x\ {\isasymsqsubseteq}\ s\ {\isasymand}\ y\ {\isasymsqsubseteq}\ s\ {\isasymand}\ {\isacharparenleft}{\kern0pt}{\isasymforall}\ z{\isachardot}{\kern0pt}\ x\ {\isasymsqsubseteq}\ z\ {\isasymand}\ y\ {\isasymsqsubseteq}\ z\ {\isasymlongrightarrow}\ s\ {\isasymsqsubseteq}\ z{\isacharparenright}{\kern0pt}{\isacharparenright}{\kern0pt}{\isachardoublequoteclose}\isanewline
\isanewline
\isacommand{theorem}\isamarkupfalse%
\ is{\isacharunderscore}{\kern0pt}inf{\isacharunderscore}{\kern0pt}uniq{\isacharcolon}{\kern0pt}\ {\isachardoublequoteopen}{\isasymlbrakk}is{\isacharunderscore}{\kern0pt}inf\ x\ y\ i{\isacharsemicolon}{\kern0pt}\ is{\isacharunderscore}{\kern0pt}inf\ x\ y\ i{\isacharprime}{\kern0pt}{\isasymrbrakk}\ {\isasymLongrightarrow}\ i\ {\isacharequal}{\kern0pt}\ i{\isacharprime}{\kern0pt}{\isachardoublequoteclose}\isanewline
%
\isadelimproof
\ \ %
\endisadelimproof
%
\isatagproof
\isacommand{oops}\isamarkupfalse%
%
\endisatagproof
{\isafoldproof}%
%
\isadelimproof
\isanewline
%
\endisadelimproof
\isacommand{theorem}\isamarkupfalse%
\ is{\isacharunderscore}{\kern0pt}sup{\isacharunderscore}{\kern0pt}uniq{\isacharcolon}{\kern0pt}\ {\isachardoublequoteopen}{\isasymlbrakk}is{\isacharunderscore}{\kern0pt}sup\ x\ y\ s{\isacharsemicolon}{\kern0pt}\ is{\isacharunderscore}{\kern0pt}sup\ x\ y\ s{\isacharprime}{\kern0pt}{\isasymrbrakk}\ {\isasymLongrightarrow}\ s\ {\isacharequal}{\kern0pt}\ s{\isacharprime}{\kern0pt}{\isachardoublequoteclose}\isanewline
%
\isadelimproof
\ \ %
\endisadelimproof
%
\isatagproof
\isacommand{oops}\isamarkupfalse%
%
\endisatagproof
{\isafoldproof}%
%
\isadelimproof
\isanewline
%
\endisadelimproof
\isanewline
\isacommand{end}\isamarkupfalse%
\isanewline
\isanewline
\isacommand{print{\isacharunderscore}{\kern0pt}locale}\isamarkupfalse%
{\isacharbang}{\kern0pt}\ partial{\isacharunderscore}{\kern0pt}order%
\isadelimdocument
%
\endisadelimdocument
%
\isatagdocument
%
\isamarkupsubsection{Import%
}
\isamarkuptrue%
%
\endisatagdocument
{\isafolddocument}%
%
\isadelimdocument
%
\endisadelimdocument
\isacommand{locale}\isamarkupfalse%
\ total{\isacharunderscore}{\kern0pt}order\ {\isacharequal}{\kern0pt}\ partial{\isacharunderscore}{\kern0pt}order\ {\isacharplus}{\kern0pt}\isanewline
\ \ \isakeyword{assumes}\ total{\isacharcolon}{\kern0pt}\ {\isachardoublequoteopen}x\ {\isasymsqsubseteq}\ y\ {\isasymor}\ y\ {\isasymsqsubseteq}\ x{\isachardoublequoteclose}\isanewline
\isanewline
\isacommand{locale}\isamarkupfalse%
\ lattice\ {\isacharequal}{\kern0pt}\ partial{\isacharunderscore}{\kern0pt}order\ {\isacharplus}{\kern0pt}\isanewline
\ \ \isakeyword{assumes}\ ex{\isacharunderscore}{\kern0pt}inf{\isacharcolon}{\kern0pt}\ {\isachardoublequoteopen}{\isasymexists}\ inf{\isachardot}{\kern0pt}\ is{\isacharunderscore}{\kern0pt}inf\ x\ y\ inf{\isachardoublequoteclose}\isanewline
\ \ \ \ \isakeyword{and}\ ex{\isacharunderscore}{\kern0pt}sup{\isacharcolon}{\kern0pt}\ {\isachardoublequoteopen}{\isasymexists}\ sup{\isachardot}{\kern0pt}\ is{\isacharunderscore}{\kern0pt}sup\ x\ y\ sup{\isachardoublequoteclose}\isanewline
\isakeyword{begin}\isanewline
\isanewline
\isacommand{definition}\isamarkupfalse%
\ meet\ {\isacharparenleft}{\kern0pt}\isakeyword{infixl}\ {\isachardoublequoteopen}{\isasymsqinter}{\isachardoublequoteclose}\ {\isadigit{7}}{\isadigit{0}}{\isacharparenright}{\kern0pt}\ \isakeyword{where}\ {\isachardoublequoteopen}x\ {\isasymsqinter}\ y\ {\isacharequal}{\kern0pt}\ {\isacharparenleft}{\kern0pt}THE\ inf{\isachardot}{\kern0pt}\ is{\isacharunderscore}{\kern0pt}inf\ x\ y\ inf{\isacharparenright}{\kern0pt}{\isachardoublequoteclose}\isanewline
\isacommand{definition}\isamarkupfalse%
\ join\ {\isacharparenleft}{\kern0pt}\isakeyword{infixl}\ {\isachardoublequoteopen}{\isasymsqunion}{\isachardoublequoteclose}\ {\isadigit{6}}{\isadigit{5}}{\isacharparenright}{\kern0pt}\ \isakeyword{where}\ {\isachardoublequoteopen}x\ {\isasymsqunion}\ y\ {\isacharequal}{\kern0pt}\ {\isacharparenleft}{\kern0pt}THE\ sup{\isachardot}{\kern0pt}\ is{\isacharunderscore}{\kern0pt}sup\ x\ y\ sup{\isacharparenright}{\kern0pt}{\isachardoublequoteclose}\isanewline
\isanewline
\isacommand{lemma}\isamarkupfalse%
\ meet{\isacharunderscore}{\kern0pt}left{\isacharcolon}{\kern0pt}\ {\isachardoublequoteopen}x\ {\isasymsqinter}\ y\ {\isasymsqsubseteq}\ x{\isachardoublequoteclose}%
\isadelimproof
\ %
\endisadelimproof
%
\isatagproof
\isacommand{oops}\isamarkupfalse%
%
\endisatagproof
{\isafoldproof}%
%
\isadelimproof
%
\endisadelimproof
\isanewline
\isanewline
\isacommand{end}\isamarkupfalse%
%
\isadelimdocument
%
\endisadelimdocument
%
\isatagdocument
%
\isamarkupsubsection{Interpretation%
}
\isamarkuptrue%
%
\endisatagdocument
{\isafolddocument}%
%
\isadelimdocument
%
\endisadelimdocument
%
\begin{isamarkuptext}%
The declaration sublocale \isa{l{\isadigit{1}}\ {\isasymsubseteq}\ l{\isadigit{2}}} causes locale l2 to be interpreted in the context of l1. 
This means that all conclusions of l2 are made available in l1.%
\end{isamarkuptext}\isamarkuptrue%
\isacommand{sublocale}\isamarkupfalse%
\ total{\isacharunderscore}{\kern0pt}order\ {\isasymsubseteq}\ lattice%
\begin{isamarkuptext}%
The sublocale command generates a goal, which must be discharged by the user:%
\end{isamarkuptext}\isamarkuptrue%
%
\isadelimproof
%
\endisadelimproof
%
\isatagproof
\isacommand{proof}\isamarkupfalse%
\ unfold{\isacharunderscore}{\kern0pt}locales\ \isanewline
\ \ \isacommand{fix}\isamarkupfalse%
\ x\ y\isanewline
\ \ \isacommand{thm}\isamarkupfalse%
\ is{\isacharunderscore}{\kern0pt}inf{\isacharunderscore}{\kern0pt}def\isanewline
\ \ \isacommand{from}\isamarkupfalse%
\ total\ \isacommand{have}\isamarkupfalse%
\ {\isachardoublequoteopen}is{\isacharunderscore}{\kern0pt}inf\ x\ y\ {\isacharparenleft}{\kern0pt}if\ x\ {\isasymsqsubseteq}\ y\ then\ x\ else\ y{\isacharparenright}{\kern0pt}{\isachardoublequoteclose}\ \isacommand{by}\isamarkupfalse%
\ {\isacharparenleft}{\kern0pt}auto\ simp{\isacharcolon}{\kern0pt}\ is{\isacharunderscore}{\kern0pt}inf{\isacharunderscore}{\kern0pt}def{\isacharparenright}{\kern0pt}\isanewline
\ \ \isacommand{thus}\isamarkupfalse%
\ {\isachardoublequoteopen}{\isasymexists}\ inf{\isachardot}{\kern0pt}\ is{\isacharunderscore}{\kern0pt}inf\ x\ y\ inf{\isachardoublequoteclose}\ \isacommand{{\isachardot}{\kern0pt}{\isachardot}{\kern0pt}}\isamarkupfalse%
\isanewline
\ \ \isacommand{from}\isamarkupfalse%
\ total\ \isacommand{have}\isamarkupfalse%
\ {\isachardoublequoteopen}is{\isacharunderscore}{\kern0pt}sup\ x\ y\ {\isacharparenleft}{\kern0pt}if\ x\ {\isasymsqsubseteq}\ y\ then\ y\ else\ x{\isacharparenright}{\kern0pt}{\isachardoublequoteclose}\ \isacommand{by}\isamarkupfalse%
\ {\isacharparenleft}{\kern0pt}auto\ simp{\isacharcolon}{\kern0pt}\ is{\isacharunderscore}{\kern0pt}sup{\isacharunderscore}{\kern0pt}def{\isacharparenright}{\kern0pt}\isanewline
\ \ \isacommand{thus}\isamarkupfalse%
\ {\isachardoublequoteopen}{\isasymexists}\ sup{\isachardot}{\kern0pt}\ is{\isacharunderscore}{\kern0pt}sup\ x\ y\ sup{\isachardoublequoteclose}\ \isacommand{{\isachardot}{\kern0pt}{\isachardot}{\kern0pt}}\isamarkupfalse%
\isanewline
\isacommand{qed}\isamarkupfalse%
%
\endisatagproof
{\isafoldproof}%
%
\isadelimproof
%
\endisadelimproof
%
\begin{isamarkuptext}%
The command interpretation is for the interpretation of locale in theories.%
\end{isamarkuptext}\isamarkuptrue%
%
\begin{isamarkuptext}%
In the following example, the parameter of locale \isa{partial{\isacharunderscore}{\kern0pt}order} is replaced
by \isa{{\isacharparenleft}{\kern0pt}{\isasymle}{\isacharparenright}{\kern0pt}} and the locale instance is interpreted in the current theory.%
\end{isamarkuptext}\isamarkuptrue%
\isacommand{interpretation}\isamarkupfalse%
\ int{\isacharcolon}{\kern0pt}\ partial{\isacharunderscore}{\kern0pt}order\ {\isachardoublequoteopen}{\isacharparenleft}{\kern0pt}{\isasymle}{\isacharparenright}{\kern0pt}\ {\isacharcolon}{\kern0pt}{\isacharcolon}{\kern0pt}\ int\ {\isasymRightarrow}\ int\ {\isasymRightarrow}\ bool{\isachardoublequoteclose}\isanewline
%
\isadelimproof
\ \ %
\endisadelimproof
%
\isatagproof
\isacommand{apply}\isamarkupfalse%
\ unfold{\isacharunderscore}{\kern0pt}locales\ \isanewline
\ \ \ \ \isacommand{apply}\isamarkupfalse%
\ auto\ \isanewline
\ \ \isacommand{done}\isamarkupfalse%
%
\endisatagproof
{\isafoldproof}%
%
\isadelimproof
\isanewline
%
\endisadelimproof
\isanewline
\isacommand{thm}\isamarkupfalse%
\ int{\isachardot}{\kern0pt}trans\isanewline
\isacommand{thm}\isamarkupfalse%
\ int{\isachardot}{\kern0pt}less{\isacharunderscore}{\kern0pt}def%
\begin{isamarkuptext}%
We want to replace int.less by \isa{{\isacharless}{\kern0pt}}.%
\end{isamarkuptext}\isamarkuptrue%
%
\begin{isamarkuptext}%
In order to allow for the desired replacement,
interpretation accepts equations in addition to the parameter instantiation. 
These follow the locale expression and are indicated with the keyword rewrites.%
\end{isamarkuptext}\isamarkuptrue%
\isacommand{interpretation}\isamarkupfalse%
\ int{\isacharcolon}{\kern0pt}\ partial{\isacharunderscore}{\kern0pt}order\ {\isachardoublequoteopen}{\isacharparenleft}{\kern0pt}{\isasymle}{\isacharparenright}{\kern0pt}\ {\isacharcolon}{\kern0pt}{\isacharcolon}{\kern0pt}\ {\isacharbrackleft}{\kern0pt}int{\isacharcomma}{\kern0pt}\ int{\isacharbrackright}{\kern0pt}\ {\isasymRightarrow}\ bool{\isachardoublequoteclose}\isanewline
\ \ \isakeyword{rewrites}\ {\isachardoublequoteopen}int{\isachardot}{\kern0pt}less\ x\ y\ {\isacharequal}{\kern0pt}\ {\isacharparenleft}{\kern0pt}x\ {\isacharless}{\kern0pt}\ y{\isacharparenright}{\kern0pt}{\isachardoublequoteclose}\isanewline
\ \ \isanewline
%
\isadelimproof
%
\endisadelimproof
%
\isatagproof
\isacommand{proof}\isamarkupfalse%
{\isacharminus}{\kern0pt}\isanewline
\ \ \isacommand{show}\isamarkupfalse%
\ {\isachardoublequoteopen}partial{\isacharunderscore}{\kern0pt}order\ {\isacharparenleft}{\kern0pt}{\isacharparenleft}{\kern0pt}{\isasymle}{\isacharparenright}{\kern0pt}\ {\isacharcolon}{\kern0pt}{\isacharcolon}{\kern0pt}\ int\ {\isasymRightarrow}\ int\ {\isasymRightarrow}\ bool{\isacharparenright}{\kern0pt}{\isachardoublequoteclose}\isanewline
\ \ \ \ \isacommand{by}\isamarkupfalse%
\ unfold{\isacharunderscore}{\kern0pt}locales\ auto\isanewline
\ \ \isacommand{show}\isamarkupfalse%
\ {\isachardoublequoteopen}partial{\isacharunderscore}{\kern0pt}order{\isachardot}{\kern0pt}less\ {\isacharparenleft}{\kern0pt}{\isasymle}{\isacharparenright}{\kern0pt}\ x\ y\ {\isacharequal}{\kern0pt}\ {\isacharparenleft}{\kern0pt}x\ {\isacharless}{\kern0pt}\ y{\isacharparenright}{\kern0pt}{\isachardoublequoteclose}\isanewline
\ \ \ \ \isacommand{unfolding}\isamarkupfalse%
\ partial{\isacharunderscore}{\kern0pt}order{\isachardot}{\kern0pt}less{\isacharunderscore}{\kern0pt}def\ {\isacharbrackleft}{\kern0pt}OF\ {\isacartoucheopen}\ partial{\isacharunderscore}{\kern0pt}order\ {\isacharparenleft}{\kern0pt}{\isasymle}{\isacharparenright}{\kern0pt}\ {\isacartoucheclose}{\isacharbrackright}{\kern0pt}\isanewline
\ \ \ \ \isacommand{by}\isamarkupfalse%
\ auto\isanewline
\isacommand{qed}\isamarkupfalse%
%
\endisatagproof
{\isafoldproof}%
%
\isadelimproof
%
\endisadelimproof
%
\begin{isamarkuptext}%
In the above example, the fact that \isa{{\isasymle}} is a partial order for the integers
was used in the second goal to discharge the premise in the definition of \isa{{\isasymsqsubseteq}}.
In general, proofs of the equations not only may involve definitions from the
interpreted locale but arbitrarily complex arguments in the context of the
locale. Therefore it would be convenient to have the interpreted locale conclusions temporarily available in the proof. 
This can be achieved by a locale interpretation in the proof body. The command for local interpretations is
interpret.%
\end{isamarkuptext}\isamarkuptrue%
\isacommand{interpretation}\isamarkupfalse%
\ int{\isacharcolon}{\kern0pt}\ partial{\isacharunderscore}{\kern0pt}order\ {\isachardoublequoteopen}{\isacharparenleft}{\kern0pt}{\isasymle}{\isacharparenright}{\kern0pt}\ {\isacharcolon}{\kern0pt}{\isacharcolon}{\kern0pt}\ int\ {\isasymRightarrow}\ int\ {\isasymRightarrow}\ bool{\isachardoublequoteclose}\isanewline
\ \ \isakeyword{rewrites}\ {\isachardoublequoteopen}int{\isachardot}{\kern0pt}less\ x\ y\ {\isacharequal}{\kern0pt}\ {\isacharparenleft}{\kern0pt}x\ {\isacharless}{\kern0pt}\ y{\isacharparenright}{\kern0pt}{\isachardoublequoteclose}\isanewline
%
\isadelimproof
%
\endisadelimproof
%
\isatagproof
\isacommand{proof}\isamarkupfalse%
\ {\isacharminus}{\kern0pt}\isanewline
\ \ \isacommand{show}\isamarkupfalse%
\ {\isachardoublequoteopen}partial{\isacharunderscore}{\kern0pt}order\ {\isacharparenleft}{\kern0pt}{\isacharparenleft}{\kern0pt}{\isasymle}{\isacharparenright}{\kern0pt}\ {\isacharcolon}{\kern0pt}{\isacharcolon}{\kern0pt}\ int\ {\isasymRightarrow}\ int\ {\isasymRightarrow}\ bool{\isacharparenright}{\kern0pt}{\isachardoublequoteclose}\isanewline
\ \ \ \ \isacommand{by}\isamarkupfalse%
\ unfold{\isacharunderscore}{\kern0pt}locales\ auto\isanewline
\ \ \isacommand{then}\isamarkupfalse%
\ \isacommand{interpret}\isamarkupfalse%
\ int{\isacharcolon}{\kern0pt}\ partial{\isacharunderscore}{\kern0pt}order\ {\isachardoublequoteopen}{\isacharparenleft}{\kern0pt}{\isasymle}{\isacharparenright}{\kern0pt}\ {\isacharcolon}{\kern0pt}{\isacharcolon}{\kern0pt}\ {\isacharbrackleft}{\kern0pt}int{\isacharcomma}{\kern0pt}\ int{\isacharbrackright}{\kern0pt}\ {\isasymRightarrow}\ bool{\isachardoublequoteclose}\ \isacommand{{\isachardot}{\kern0pt}}\isamarkupfalse%
\isanewline
\ \ \isacommand{show}\isamarkupfalse%
\ {\isachardoublequoteopen}int{\isachardot}{\kern0pt}less\ x\ y\ {\isacharequal}{\kern0pt}\ {\isacharparenleft}{\kern0pt}x\ {\isacharless}{\kern0pt}\ y{\isacharparenright}{\kern0pt}{\isachardoublequoteclose}\isanewline
\ \ \ \ \isacommand{unfolding}\isamarkupfalse%
\ int{\isachardot}{\kern0pt}less{\isacharunderscore}{\kern0pt}def\ \isacommand{by}\isamarkupfalse%
\ auto\isanewline
\isacommand{qed}\isamarkupfalse%
%
\endisatagproof
{\isafoldproof}%
%
\isadelimproof
%
\endisadelimproof
%
\begin{isamarkuptext}%
theorems from the local interpretation disappear after leaving the proof context — that is,
after the succeeding next or qed statement%
\end{isamarkuptext}\isamarkuptrue%
%
\isadelimtheory
%
\endisadelimtheory
%
\isatagtheory
\isacommand{end}\isamarkupfalse%
%
\endisatagtheory
{\isafoldtheory}%
%
\isadelimtheory
%
\endisadelimtheory
%
\end{isabellebody}%
\endinput
%:%file=Locales.tex%:%
%:%10=1%:%
%:%11=1%:%
%:%12=2%:%
%:%13=3%:%
%:%27=5%:%
%:%39=8%:%
%:%40=9%:%
%:%42=12%:%
%:%43=12%:%
%:%44=13%:%
%:%45=14%:%
%:%46=15%:%
%:%47=16%:%
%:%49=19%:%
%:%50=20%:%
%:%51=21%:%
%:%53=24%:%
%:%54=24%:%
%:%55=25%:%
%:%56=26%:%
%:%57=26%:%
%:%59=29%:%
%:%60=30%:%
%:%61=31%:%
%:%62=32%:%
%:%64=35%:%
%:%65=35%:%
%:%66=36%:%
%:%67=36%:%
%:%74=38%:%
%:%84=40%:%
%:%85=40%:%
%:%86=41%:%
%:%87=42%:%
%:%89=45%:%
%:%91=47%:%
%:%92=47%:%
%:%94=49%:%
%:%95=50%:%
%:%97=52%:%
%:%98=52%:%
%:%99=53%:%
%:%100=54%:%
%:%101=54%:%
%:%102=55%:%
%:%105=56%:%
%:%109=56%:%
%:%110=56%:%
%:%111=56%:%
%:%125=58%:%
%:%137=60%:%
%:%139=62%:%
%:%140=62%:%
%:%141=63%:%
%:%142=64%:%
%:%143=65%:%
%:%144=66%:%
%:%145=66%:%
%:%146=67%:%
%:%147=68%:%
%:%148=69%:%
%:%149=70%:%
%:%150=70%:%
%:%151=71%:%
%:%152=72%:%
%:%153=73%:%
%:%154=74%:%
%:%155=74%:%
%:%158=75%:%
%:%162=75%:%
%:%168=75%:%
%:%171=76%:%
%:%172=76%:%
%:%175=77%:%
%:%179=77%:%
%:%185=77%:%
%:%188=78%:%
%:%189=79%:%
%:%190=79%:%
%:%191=80%:%
%:%192=81%:%
%:%193=81%:%
%:%200=83%:%
%:%210=85%:%
%:%211=85%:%
%:%212=86%:%
%:%213=87%:%
%:%214=88%:%
%:%215=88%:%
%:%216=89%:%
%:%217=90%:%
%:%218=91%:%
%:%219=92%:%
%:%220=93%:%
%:%221=93%:%
%:%222=94%:%
%:%223=94%:%
%:%224=95%:%
%:%225=96%:%
%:%226=96%:%
%:%228=96%:%
%:%232=96%:%
%:%240=96%:%
%:%241=97%:%
%:%242=98%:%
%:%250=100%:%
%:%262=103%:%
%:%263=104%:%
%:%265=106%:%
%:%266=106%:%
%:%268=107%:%
%:%276=108%:%
%:%277=108%:%
%:%278=109%:%
%:%279=109%:%
%:%280=110%:%
%:%281=110%:%
%:%282=111%:%
%:%283=111%:%
%:%284=111%:%
%:%285=111%:%
%:%286=112%:%
%:%287=112%:%
%:%288=112%:%
%:%289=113%:%
%:%290=113%:%
%:%291=113%:%
%:%292=113%:%
%:%293=114%:%
%:%294=114%:%
%:%295=114%:%
%:%296=115%:%
%:%306=119%:%
%:%310=121%:%
%:%311=122%:%
%:%313=125%:%
%:%314=125%:%
%:%317=126%:%
%:%321=126%:%
%:%322=126%:%
%:%323=127%:%
%:%324=127%:%
%:%325=128%:%
%:%331=128%:%
%:%334=129%:%
%:%335=130%:%
%:%336=130%:%
%:%337=131%:%
%:%338=131%:%
%:%340=134%:%
%:%344=138%:%
%:%345=139%:%
%:%346=140%:%
%:%348=143%:%
%:%349=143%:%
%:%350=144%:%
%:%351=145%:%
%:%351=146%:%
%:%358=147%:%
%:%359=147%:%
%:%360=148%:%
%:%361=148%:%
%:%362=149%:%
%:%363=149%:%
%:%364=150%:%
%:%365=150%:%
%:%366=151%:%
%:%367=151%:%
%:%368=152%:%
%:%369=152%:%
%:%370=153%:%
%:%380=156%:%
%:%381=157%:%
%:%382=158%:%
%:%383=159%:%
%:%384=160%:%
%:%385=161%:%
%:%386=162%:%
%:%388=165%:%
%:%389=165%:%
%:%390=166%:%
%:%397=167%:%
%:%398=167%:%
%:%399=168%:%
%:%400=168%:%
%:%401=169%:%
%:%402=169%:%
%:%403=170%:%
%:%404=170%:%
%:%405=170%:%
%:%406=170%:%
%:%407=171%:%
%:%408=171%:%
%:%409=172%:%
%:%410=172%:%
%:%411=172%:%
%:%412=173%:%
%:%422=176%:%
%:%423=177%:%
%:%431=180%:%
